\documentclass[
	% -- opções da classe memoir --
	article,			% indica que é um artigo acadêmico
	11pt,				% tamanho da fonte
	oneside,			% para impressão apenas no recto. Oposto a twoside
	a4paper,			% tamanho do papel. 
	% -- opções da classe abntex2 --
	%chapter=TITLE,		% títulos de capítulos convertidos em letras maiúsculas
	%section=TITLE,		% títulos de seções convertidos em letras maiúsculas
	%subsection=TITLE,	% títulos de subseções convertidos em letras maiúsculas
	%subsubsection=TITLE % títulos de subsubseções convertidos em letras maiúsculas
	% -- opções do pacote babel --
	english,			% idioma adicional para hifenização
	brazil,				% o último idioma é o principal do documento
	sumario=tradicional
	]{abntex2}


% ---
% PACOTES
% ---

% ---
% Pacotes fundamentais 
% ---
\usepackage{lmodern}			% Usa a fonte Latin Modern
\usepackage[T1]{fontenc}		% Selecao de codigos de fonte.
\usepackage[utf8]{inputenc}		% Codificacao do documento (conversão automática dos acentos)
\usepackage{indentfirst}		% Indenta o primeiro parágrafo de cada seção.
\usepackage{nomencl} 			% Lista de simbolos
\usepackage{color}				% Controle das cores
\usepackage{graphicx}			% Inclusão de gráficos
\usepackage{microtype} 			% para melhorias de justificação
% ---
		
% ---
% Pacotes adicionais, usados apenas no âmbito do Modelo Canônico do abnteX2
% ---
\usepackage{lipsum}				% para geração de dummy text
% ---
		
% ---
% Pacotes de citações
% ---
\usepackage[brazilian,hyperpageref]{backref}	 % Paginas com as citações na bibl
\usepackage[alf]{abntex2cite}	% Citações padrão ABNT

\usepackage{backnaur} %  especificação da grmática
\usepackage{syntax} % especificação gramática
\usepackage{subfiles} % modularização em arquivos
\usepackage{hyperref} % hyper-links

%%%%%%%%%%%%%%%%%%%
%%%%%%%%%%%%%%%%%%% https://tex.stackexchange.com/questions/348651/c-code-to-add-in-the-document
%%%%%%%%%%%%%%%%%%%
\usepackage{xcolor}
\usepackage{listings}

\definecolor{mGreen}{rgb}{0,0.6,0}
\definecolor{mGray}{rgb}{0.5,0.5,0.5}
\definecolor{mPurple}{rgb}{0.58,0,0.82}
\definecolor{backgroundColour}{rgb}{0.95,0.95,0.92}

\lstdefinestyle{CStyle}{
	backgroundcolor=\color{backgroundColour},   
	commentstyle=\color{mGreen},
	keywordstyle=\color{magenta},
	numberstyle=\tiny\color{mGray},
	stringstyle=\color{mPurple},
	basicstyle=\footnotesize,
	breakatwhitespace=false,         
	breaklines=true,                 
	captionpos=b,                    
	keepspaces=true,                 
	numbers=left,                    
	numbersep=5pt,                  
	showspaces=false,                
	showstringspaces=false,
	showtabs=false,                  
	tabsize=2,
	language=C
}
%%%%%%%%%%%%%%%%%%%
%%%%%%%%%%%%%%%%%%%
%%%%%%%%%%%%%%%%%%%

% ---

% ---
% Configurações do pacote backref
% Usado sem a opção hyperpageref de backref

\renewcommand{\it}[1]{\textit{#1}}
\renewcommand{\bf}[1]{\textbf{#1}}

\renewcommand{\backrefpagesname}{Citado na(s) página(s):~}
% Texto padrão antes do número das páginas
\renewcommand{\backref}{}
% Define os textos da citação
\renewcommand*{\backrefalt}[4]{
	\ifcase #1 %
		Nenhuma citação no texto.%
	\or
		Citado na página #2.%
	\else
		Citado #1 vezes nas páginas #2.%
	\fi}%
% ---

% --- Informações de dados para CAPA e FOLHA DE ROSTO ---
\titulo{C-minus com operações matriciais}
%\tituloestrangeiro{Canonical article template in \abnTeX: optional foreign title}

\autor{
Leonardo Maffei da Silva\thanks{leoitu22hotmail.com@gmail.com. \url{https://www.linkedin.com/in/leonardo-maffei-ti/}} }

\local{Brasil}
\data{2019}
% ---

% ---
% Configurações de aparência do PDF final

% alterando o aspecto da cor azul
\definecolor{blue}{RGB}{41,5,195}

% informações do PDF
\makeatletter
\hypersetup{
     	%pagebackref=true,
		pdftitle={\@title}, 
		pdfauthor={\@author},
    	pdfsubject={Modelo de artigo científico com abnTeX2},
	    pdfcreator={LaTeX with abnTeX2},
		pdfkeywords={abnt}{latex}{abntex}{abntex2}{atigo científico}, 
		colorlinks=true,       		% false: boxed links; true: colored links
    	linkcolor=blue,          	% color of internal links
    	citecolor=blue,        		% color of links to bibliography
    	filecolor=magenta,      		% color of file links
		urlcolor=blue,
		bookmarksdepth=4
}
\makeatother
% --- 

% ---
% compila o indice
% ---
\makeindex
% ---

% ---
% Altera as margens padrões
% ---
\setlrmarginsandblock{3cm}{3cm}{*}
\setulmarginsandblock{3cm}{3cm}{*}
\checkandfixthelayout
% ---

% --- 
% Espaçamentos entre linhas e parágrafos 
% --- 

% O tamanho do parágrafo é dado por:
\setlength{\parindent}{1.3cm}

% Controle do espaçamento entre um parágrafo e outro:
\setlength{\parskip}{0.2cm}  % tente também \onelineskip

% Espaçamento simples
\SingleSpacing


% ----
% Início do documento
% ----
\begin{document}

% Seleciona o idioma do documento (conforme pacotes do babel)
%\selectlanguage{english}
\selectlanguage{brazil}

% Retira espaço extra obsoleto entre as frases.
\frenchspacing 

% ----------------------------------------------------------
% ELEMENTOS PRÉ-TEXTUAIS
% ----------------------------------------------------------

%---
%
% Se desejar escrever o artigo em duas colunas, descomente a linha abaixo
% e a linha com o texto ``FIM DE ARTIGO EM DUAS COLUNAS''.
% \twocolumn[    		% INICIO DE ARTIGO EM DUAS COLUNAS
%
%---

% página de titulo principal (obrigatório)
\maketitle


% titulo em outro idioma (opcional)



% resumo em português
\begin{resumoumacoluna}
 Neste documento encontra-se a especificação da gramática de uma linguagem que estende uma versão mínima do C (conhecida por \textit{C-minus}). A nova linguagem proposta apresenta como primitivo o tipo matriz, bem como a realização das operações aritméticas básicas e outras operações mais complexas tais como potenciação e escalonamento, sendo a última realizada por função da linguagem. Grande parte da gramática foi diretamente tirada de \cite{gramatica}.
 \vspace{\onelineskip}
 
 \noindent
 \textbf{Palavras-chave}: C, linguagem, matriz, primitiva.
\end{resumoumacoluna}




\newcommand{\terminal}[1]{ \bnfpn{\textbf{#1}} }

\newcommand{\production}[1]{\bnfpn{\textit{#1}}}
\newcommand{\IT}[1]{\textit{#1}}

%\setlength{\grammarparsep}{20pt plus 1pt minus 1pt} % increase separation between rules
%\setlength{\grammarindent}{12em} % increase separation between LHS/RHS 

% ----------------------------------------------------------
% Introdução
% ----------------------------------------------------------
\section{Introdução}
Implementar-se-á, até a versão final deste artigo, um compilador para a linguagem proposta. Para sua realização, serão utilizados os conhecimentos adquiridos na disciplina \textit{Tradutores}, ministrada pela professora \hyperref{http://lattes.cnpq.br/7793795625581127}{}{}{Cláudia Nalon.}

\section{Usuário característico}
Destina-se ao estudante de álgebra linear, o qual pode usar a linguagem para, por exmeplo, confirmar se sua resolução de um sisema linear encontra-se correta, tudo isso de maneira rápida, eficiente e \it{offline}.

\section{Motivação}
Durante a realização do curso de Cálculo Numérico, o grupo do autor notou a ausência dessa \it{feature} na linguagem C. Desse modo, foi necessária a simulação desse tipo de dados, à época implementada por meio de inúmeras funções. Se houvese um tipo nativo para matriz bem como operações elementares sobre seus elementos, teria sido de grande auxílio à codificação dos diversos métodos numéricos requeridos pela disciplina.
\par

% ----------------------------------------------------------
% Seção de explicações
% ----------------------------------------------------------


\section{Gramática}
A seguir, encontra-se a gramática da linguagem proposta:
%% BEGIN GRAMÁTICA
	\subfile{grammar.tex}
%% END GRMÁTICA
Os \textit{tokens} da linguagem são: '\bf{,}', '\bf{.}', '\bf{>=}', '\bf{>}', '\bf{<}', '\bf{<=}', '\bf{==}', '\bf{!=}', '\bf{int}', '\bf{float}', '\bf{while}', '\bf{return}', '\bf{if}', '\bf{else}', '\bf{mat}', '\bf{EOL}', '\bf{+}', '\bf{-}', '\bf{*}', '\bf{/}', '\bf{\&\&}', '\bf{\^}', '\bf{||}', '\bf{!}', '\bf{\&}', '\bf{;}', '\bf{(}', '\bf{)}', '\bf{[}', '\bf{]}', '\bf{{}', '\bf{}}' e '\bf{//}',  sendo este último um caso especial por delimitar comentários de uma linha. \it{EOL} simboliza o final de uma linha qualquer, e as seguintes palavras são reservadas e portanto não podem ser utilizadas como identificadores: \bf{else}, \bf{float}, \bf{if}, \bf{int}, \bf{mat} e \bf{return}. A regra de produção \it{ASCII} produz qualquer símbolo contido na tabela \it{ascii}. Por fim, não constam na gramática, mas as funções  \it{scan} e \it{print} fazem parte da linguagem, sendo responsáveis pelas operações de leitura e impressão na saída padrão do ambiente de execução do programa (usualmente, o console).

\section{Semântica}
A semântica da linguagem é quase idêntica à da linguagem C: declarações de variável (a menos das do tipo \bf{mat}), funções e expressões têm semãntica similar. Sendo esta linguagem uma estensão de um subconjunto da linguagem \it{C}, a principal diferença está no tipo de dados \bf{mat} (abreviação de \it{matrix}, "matriz" em inglês). Esse tipo de dados é similar aos \it{arrays} em C, porém limitado do ponto de vista da composição pois não é possível a criação de matrizes aninhadas. Entretando, é possível a realização das quatro operações aritméticas básicas diretamente com matrizes, bem como a realização de potenciação de matrizes de forma \it{rápida} e algumas operações sobre elas, como resolução de sistemas lineares e escalonamento. Tais operações mais complexas estarão presentes por meio de funções padrão da linguagem.

\section{Exemplo de programa na linguagem}
A seguir, trechos de código pertencente à nova linguagem.

\begin{lstlisting}[style=CStyle]
int foo(void);
mat<int> matriz[3][4];	// Criacao de um sistema 
matrix = {
	{1, 2, 3, 4},
	{4, 3 ,2, 1},
	{-1, -2, -3, -4},
	{45, 87, 9, 18}
};

int main(){
	int x;
	mat<float> casted = matriz + matriz;
	scan(x);
	print(x); 
}
\end{lstlisting}

\section{Agradecimentos}
Agradecimentos ao autor anônimo responsável por \cite{gramatica}, sem o qual este trabalho teria sido muito mais custoso, bem como ao usuário \hyperref{https://tex.stackexchange.com/users/42366/croco}{}{}{CroCro}, o qual possibilitou conforme \cite{custom} rápida customização do código-exemplo da nova linguagem. Por fim, o pacote utilizado para redação da gramática foi sugerido pelo usuário \hyperref{https://tex.stackexchange.com/questions/348651/c-code-to-add-in-the-document}{}{}{AlanMunn}, conforme \cite{bnf}.

% ---
% Finaliza a parte no bookmark do PDF, para que se inicie o bookmark na raiz
% ---
\bookmarksetup{startatroot}% 
% ---

% ---
% Conclusão
% ---
\postextual

\bibliography{bib}

\end{document}

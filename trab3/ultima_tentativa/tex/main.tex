\documentclass[
	% -- opções da classe memoir --
	article,			% indica que é um artigo acadêmico
	11pt,				% tamanho da fonte
	oneside,			% para impressão apenas no recto. Oposto a twoside
	a4paper,			% tamanho do papel. 
	% -- opções da classe abntex2 --
	%chapter=TITLE,		% títulos de capítulos convertidos em letras maiúsculas
	%section=TITLE,		% títulos de seções convertidos em letras maiúsculas
	%subsection=TITLE,	% títulos de subseções convertidos em letras maiúsculas
	%subsubsection=TITLE % títulos de subsubseções convertidos em letras maiúsculas
	% -- opções do pacote babel --
	english,			% idioma adicional para hifenização
	brazil,				% o último idioma é o principal do documento
	sumario=tradicional
	]{abntex2}


% ---
% PACOTES
% ---

% ---
% Pacotes fundamentais 
% ---
\usepackage{lmodern}			% Usa a fonte Latin Modern
\usepackage[T1]{fontenc}		% Selecao de codigos de fonte.
\usepackage[utf8]{inputenc}		% Codificacao do documento (conversão automática dos acentos)
\usepackage{indentfirst}		% Indenta o primeiro parágrafo de cada seção.
\usepackage{nomencl} 			% Lista de simbolos
\usepackage{color}				% Controle das cores
\usepackage{graphicx}			% Inclusão de gráficos
\usepackage{microtype} 			% para melhorias de justificação
% ---
		
% ---
% Pacotes adicionais, usados apenas no âmbito do Modelo Canônico do abnteX2
% ---
\usepackage{lipsum}				% para geração de dummy text
% ---
		
% ---
% Pacotes de citações
% ---
\usepackage[brazilian,hyperpageref]{backref}	 % Paginas com as citações na bibl
\usepackage[alf]{abntex2cite}	% Citações padrão ABNT

\usepackage{backnaur} %  especificação da grmática
\usepackage{syntax} % especificação gramática
\usepackage{subfiles} % modularização em arquivos
\usepackage{hyperref} % hyper-links

%%%%%%%%%%%%%%%%%%%
%%%%%%%%%%%%%%%%%%% https://tex.stackexchange.com/questions/348651/c-code-to-add-in-the-document
%%%%%%%%%%%%%%%%%%%
\usepackage{xcolor}
\usepackage{listings}

\lstset{
	basicstyle=\ttfamily,
	xleftmargin=3em,
	literate={->}{$\rightarrow$}{2}
	{α}{$\alpha$}{1}
	{δ}{$\delta$}{1}
}

\definecolor{mGreen}{rgb}{0,0.6,0}
\definecolor{mGray}{rgb}{0.5,0.5,0.5}
\definecolor{mPurple}{rgb}{0.58,0,0.82}
\definecolor{backgroundColour}{rgb}{0.95,0.95,0.92}

\lstdefinestyle{CStyle}{
	backgroundcolor=\color{backgroundColour},   
	commentstyle=\color{mGreen},
	keywordstyle=\color{magenta},
	numberstyle=\tiny\color{mGray},
	stringstyle=\color{mPurple},
	basicstyle=\footnotesize,
	breakatwhitespace=false,         
	breaklines=true,                 
	captionpos=b,                    
	keepspaces=true,                 
	numbers=left,                    
	numbersep=5pt,                  
	showspaces=false,                
	showstringspaces=false,
	showtabs=false,                  
	tabsize=2,
	language=C
}
%%%%%%%%%%%%%%%%%%%
%%%%%%%%%%%%%%%%%%%
%%%%%%%%%%%%%%%%%%%

% ---

% ---
% Configurações do pacote backref
% Usado sem a opção hyperpageref de backref

\renewcommand{\it}[1]{\textit{#1}}
\renewcommand{\bf}[1]{\textbf{#1}}

\renewcommand{\backrefpagesname}{Citado na(s) página(s):~}
% Texto padrão antes do número das páginas
\renewcommand{\backref}{}
% Define os textos da citação
\renewcommand*{\backrefalt}[4]{
	\ifcase #1 %
		Nenhuma citação no texto.%
	\or
		Citado na página #2.%
	\else
		Citado #1 vezes nas páginas #2.%
	\fi}%
% ---

% --- Informações de dados para CAPA e FOLHA DE ROSTO ---
\titulo{C-minus com operações matriciais}
%\tituloestrangeiro{Canonical article template in \abnTeX: optional foreign title}

\autor{
Leonardo Maffei da Silva\thanks{leoitu22hotmail.com@gmail.com. \url{https://www.linkedin.com/in/leonardo-maffei-ti/}} }

\local{Brasil}
\data{2019}
% ---

% ---
% Configurações de aparência do PDF final

% alterando o aspecto da cor azul
\definecolor{blue}{RGB}{41,5,195}

% informações do PDF
\makeatletter
\hypersetup{
     	%pagebackref=true,
		pdftitle={\@title}, 
		pdfauthor={\@author},
    	pdfsubject={Modelo de artigo científico com abnTeX2},
	    pdfcreator={LaTeX with abnTeX2},
		pdfkeywords={abnt}{latex}{abntex}{abntex2}{atigo científico}, 
		colorlinks=true,       		% false: boxed links; true: colored links
    	linkcolor=blue,          	% color of internal links
    	citecolor=blue,        		% color of links to bibliography
    	filecolor=magenta,      		% color of file links
		urlcolor=blue,
		bookmarksdepth=4
}
\makeatother
% --- 

% ---
% compila o indice
% ---
\makeindex
% ---

% ---
% Altera as margens padrões
% ---
\setlrmarginsandblock{3cm}{3cm}{*}
\setulmarginsandblock{3cm}{3cm}{*}
\checkandfixthelayout
% ---

% --- 
% Espaçamentos entre linhas e parágrafos 
% --- 

% O tamanho do parágrafo é dado por:
\setlength{\parindent}{1.3cm}

% Controle do espaçamento entre um parágrafo e outro:
\setlength{\parskip}{0.2cm}  % tente também \onelineskip

% Espaçamento simples
\SingleSpacing


% ----
% Início do documento
% ----
\begin{document}

% Seleciona o idioma do documento (conforme pacotes do babel)
%\selectlanguage{english}
\selectlanguage{brazil}

% Retira espaço extra obsoleto entre as frases.
\frenchspacing 

% ----------------------------------------------------------
% ELEMENTOS PRÉ-TEXTUAIS
% ----------------------------------------------------------

%---
%
% Se desejar escrever o artigo em duas colunas, descomente a linha abaixo
% e a linha com o texto ``FIM DE ARTIGO EM DUAS COLUNAS''.
% \twocolumn[    		% INICIO DE ARTIGO EM DUAS COLUNAS
%
%---

% página de titulo principal (obrigatório)
\maketitle


% titulo em outro idioma (opcional)



% resumo em português
\begin{resumoumacoluna}
 Este documento atende os fins de documentação da terceira
 parte do projeto final da disciplina \textit{Tradutores},
 ministrada pela professora Dr.a Cláudia Nalon, no segundo semestre de 2019, na Universidade de Brasília. Tal artefato descreve um pouco da implementação do analisador sintático, dificuldades encontradas durante tal processo; a nova gramática proposta em virtude de utilização de fonte anônima para construção das gramáticas anteriores; políticas de tratamento de erros e arquivos de teste para o analisador produzido.
 \vspace{\onelineskip}
 
 \noindent
 \textbf{Palavras-chave}: C, linguagem, matriz, primitiva.
\end{resumoumacoluna}




\newcommand{\terminal}[1]{ \bnfpn{\textbf{#1}} }

\newcommand{\production}[1]{\bnfpn{\textit{#1}}}
\newcommand{\IT}[1]{\textit{#1}}
\newcommand{\BF}[1]{\textbf{#1}}

%\setlength{\grammarparsep}{20pt plus 1pt minus 1pt} % increase separation between rules
%\setlength{\grammarindent}{12em} % increase separation between LHS/RHS 

\section{Introdução}
Implementar-se-á, até a versão final deste artigo, um compilador para a linguagem proposta. Para sua realização, serão utilizados os conhecimentos adquiridos na disciplina \textit{Tradutores}, ministrada pela professora \hyperref{http://lattes.cnpq.br/7793795625581127}{}{}{Cláudia Nalon.}

\section{Usuário característico}
Destina-se ao estudante de álgebra linear, o qual pode usar a linguagem para, por exmeplo, confirmar se sua resolução de um sisema linear encontra-se correta, tudo isso de maneira rápida e eficiente.

\section{Motivação}
Durante a realização do curso de Cálculo Numérico, o grupo do autor notou a ausência dessa \it{feature} na linguagem C. Desse modo, foi necessária a simulação desse tipo de dados, à época implementada por meio de inúmeras funções. Se houvese um tipo nativo para matriz bem como operações elementares sobre seus elementos, teria sido de grande auxílio à codificação dos diversos métodos numéricos requeridos pela disciplina.
\par


\section{Gramática}
Para esta entrega a gramática foi completamente reformulada. Sua especificação encontra-se abaixo:
%% BEGIN GRAMÁTICA
	\subfile{grammar.tex}
%% END GRMÁTICA
A nova gramática gera linguagem semelhante à anteriormente gerada, sendo que as principais diferenças ficam por conta da passagem de matriz como parametro, adição da palavra reservada \BF{void} para quando se for declarar uma função sem parâmetro algum, remoção de alguns operadores (por exemplo, o '-' unário) e adição de alguns erros de provável ocorrência (por exemplo, o \IT{não fechamento de parêntese} do \BF{if}).

As palavras reservadas da linguagem são: \BF{void}, \BF{if}, \BF{else}, \BF{while},  \BF{int}, \BF{float}, \BF{char}, \BF{return}, \BF{mat}, \BF{ahead}, \BF{READ} e \BF{PRINT}.
\BF{void} deve ser usado quando declarando/defininf/chamando uma função que não recebe parâmetro algum; \BF{mat} indica a declaração de um dado do tipo matriz; \BF{ahead} é utilizado para declaração de função; \BF{READ} e \BF{PRINT} são os comandos da linguagem para leitura e escrita.


% ----------------------------------------------------------
% Introdução
% ----------------------------------------------------------


% ----------------------------------------------------------
% Seção de explicações
% ----------------------------------------------------------




\section{Semântica}
A semântica da linguagem é semelhante à da linguagem C: declarações de variáveis (a menos das do tipo \bf{mat}), funções e expressões têm semântica similar. Sendo esta linguagem uma extensão de um subconjunto da linguagem \it{C}, a principal diferença está no tipo de dados \bf{mat} (abreviação de \it{matrix}). Esse tipo de dados é similar aos \it{arrays} em C, porém limitado do ponto de vista da composição pois não é possível a criação de matrizes aninhadas, nem de matrizes de vetores. Entretando, é possível a realização das quatro operações aritméticas básicas diretamente com matrizes, bem como a realização de potenciação de matrizes de forma \it{rápida} e algumas operações sobre elas, como resolução de sistemas lineares e escalonamento. Multiplicação e potenciação de matrizes são respectivamente expressas pelos novos operadores \IT{@} e \IT{@@}. Finalmente, a linguagem apena aceita passagem de parâmetros po referência e toda variável é indexável.

\section{Exemplo de programa na linguagem}
A seguir, trechos de código pertencente à nova linguagem.

\begin{lstlisting}[style=CStyle]
int main() {
float a = 10.1;
float c = 10.;

mat int m[3][3] = [
	{1, 0, 0} 
	{0 ,1, 0} 
	, 0, 1}
];
READ(a);
}'a'; '\n'; '\r'; '\\';
\end{lstlisting}

\begin{lstlisting}[style=CStyle]
float main;
{1,2,3};
"Pode ir, tudo bem..."
/* Lucero mto bom */
print('h');
print('e');
print('l');
print('l');
\end{lstlisting}

\section{Exemplo de programa não pertencentes à linguagem}

\begin{lstlisting}[style=CStyle]
" string sem dim!

int main() $$ {
;;;
}
\end{lstlisting}

\begin{lstlisting}[style=CStyle]
float chr(void);
'\\\\';
/*
int main() {

}
print("e agora, joseh?")
Obs: a exemplo do tratamento proporcionado pelo gcc,
nao eh feito nenhuma tentativa de recuperacao para
erros de comentarios sem fechamento ;)
\end{lstlisting}



\section{Implementação}
O presente analisador necessita, além dos arquivos citados abaixo
 léxico tem como dependências, além do programa \IT{flex}, os respectivos arquivos fonte, além de seus respectivos cabeçalhos (com exceção do último, por tratar do código fonte utilizado pelo \IT{flex} para geração do arquivo \textbf{lex.yy.c}):
	\begin{itemize}{
		\item{ShowTree.c}
		\item{Function.c}
		\item{Prints.c}
		\item{grammay.y}
		\item{lexico.l}
	}\end{itemize}
\begin{lstlisting}[language=bash,caption={bash version}]
	make clean
	flex leo.l
	make
	./lexico <caminho-para-arquivo>
\end{lstlisting}

\subsection{Sintático}


\subsection{Funcionamento}
\subsubsection{Léxico}
Após gerado o analisador léxico seguindo os passos descritos no início desta seção, sua utilização é bastante simples: execute o programa passando como argumento o caminho para o arquivo que deve ser aberto e processado pelo léxico. O programa então lê sequenciamente o arquivo caractere a caractere e vai exibindo os \textit{tokens} presentes no arquivo apontado à medida que os encontra, bem como trata os erros descritos na subseção logo abaixo.
A saída é exibida na saída padrão (usualmente, um console) e de forma \textit{colorida}. Entende-se que isso facilita a visualização por parte dos humanod; contudo, isso deve ser adaptado na próxima fase para que seja gerado um arquivo estruturado contendo todos os tokens identificados, o qual será utilizado pelo analisador sintático.


\subsection{Tratamento de Erros}
\subsubsection{Léxico}
O analisador léxico desenvolvido é capaz de detectar os seguintes erros:
\begin{enumerate}{
	\item \label{itm:er1}\IT{string} sem fechamento
	\item \label{itm:er2}comentários em bloco sem fechamento
	\item \label{itm:er3}caracteres não pertencentes à linguagem (individualmente)
}\end{enumerate}
O erro \ref{itm:er1} é tratado considerando que o usuário termina a \IT{string} ao final da linha, visto que não são permitidas \IT{strings} multi-linhas, e continuando a análise como se não houvesse erro.
O próximo erro (\ref{itm:er2}) não é exatamente tratado; a abordagem utilizada é a mesma do compilador \href{https://gcc.gnu.org/}{gcc 7.7.0}: é emitido um aviso ao usuário informando-lhe linha e coluna onde se inicia o comentário não finalizado.
Por fim, o último erro (\ref{itm:er3}) é tratado informando ao usuário as ocorrências desses caracteres, porém sem entrar em modo de pânico.


\subsubsection{Sintático}
O analisador produzido reporta alguns erros comuns, quais sejam:

\begin{itemize}{
		\item não inserção de ponto e vírgula após o valor a ser retornado pelo \IT{return} e o não fechamento de parêntese 
		\item expressão vazia (não são permitidas expressões lugal algum)
		\item não inserção de parêntese direito relativo à condição do \IT{if}
	}
\end{itemize}
Em caso de outros erros, a análise simplesmente é abortada. Em caso dos erros acima, a árvore sintática é montada naturalmente pelo \IT{bison}.
\subsection {Dificuldades Encontradas}
\subsubsection{Léxico}
Uma das dificuldades foi sem dúvida a familiarização com a ferramente \IT{flex}, a qual demonstrou-se muito competente porém não tão intuitiva, bem como alguns trechos de seu \href{ftp://ftp.gnu.org/old-gnu/Manuals/flex-2.5.4/html_mono/flex.html}{manual}. Além disso, surgiram dúvidas a respeito do que seriam erros léxicos, como por exemplo qual o escopo dessa categoria de erros.

Além das dúvidas acima, foram encontrados obstáculos na automação da geração do analisador por meio da ferramenta \IT{make}, questões sobre como deveria ser exibida a sequência de \IT{tokens} (quais possíveis metadados deveriam ser exebidos?) e algumas pontualidades acerca de recursão na gramática.

Em termos de codificação, além do problema recém-citado, não foi possível implementar a tabela de símbolos a tempo da entrega (ocorreram problemas de vazamento de memória e alocação equivocada). Embora esta não seja fundamental para a exibição dos tokens lidos em sequência, sua presença certamente enriqueceria o presente trabalho.

\subsubsection{Sintático}
Este trabalho exigiu esforço de codificação e pesquisa muito acima do esperado. As dificuldades principais sao listadas a seguir:
\begin{itemize}{
	\item {integração entre bison e flex, em especial relativo ao entendimento das interações entre as seções e de suas relevâncias}
	\item {decidir quais atributos deveriam estar na tabela de símbolos}
	\item{exibição da árvore (não foi possível fazê-lo; em algum ponto ocorre falha de segmentação)}
	\item{projeto da gramática: a presente gramática foi feita do zero; contudo, contém ainda alguns erros e possui um número absurdo de regras, em relação às gramáticas de muitos dos outros aluno}
	\item{volume de codificação ascrescido de repetição: muito do código feito segue um padrão muito bem definido, basicamente um modelo. Digitar tudo isso foi bastante tedioso e não parece ter contribuído para a geração de conhecimento relativo à teoria abordada nas aulas ou sobre a prática da implementação de compiladores}
	\item {passagem de dados do \IT{flex} para o \IT{bison} }
}
\end{itemize}
O principal problema encontrado, no sentido de ter tomado muito tempo de codificação, foi a questão da árvore sintática. Tentou-se fazer uso das variáveis utilizadas pelo \IT{bison} para, ao final da análise sintática, gerar a imagem da árvore de derivação. Entretando, erros acometidos durante a fase de codificação ou estruturação da solução para exibir a árvore impossibilitaram sua realização. Tal impossibilidade deu-se por conta de falha de segmentação a qual ocorre em uma das primeiras funções chamadas para exibir a árvore de \IT{parsing} \BF{após sua completa montagem (implícita)}.

A construção da árvore foi feita por meio do padrão 
\begin{lstlisting}[style=CStyle]

head : body {
	$$ = make_Head($1);
}
\end{lstlisting}
onde para cada corpo de regra há uma função \IT{make\_<NOME>} associada. Há também um tipo de dados específico para cada cabeça de regra, bem como cada tipo desses possui uma union a qual armazena structs cujo conteúdo são outros dados associados aos tipos de cada nó. Por exemplo:

\begin{lstlisting}[style=CStyle]

head : body {
$$ = make_Head($1);
}
\end{lstlisting}


\subsection{Arquivos de teste}
Os arquivos de teste encontram-se na pasta \IT{test}. Os arquivos cujo texto pertence (ou deveria pertencer) à linguagem gerada pela gramática estão prefixados pela letra \BF{c} e os que não pertencem à ela são prefixados pela letra \BF{e}. Há também arquivos de cujo nome é prefixado com a letra \BF{t}, os quais representam sequências não pertencentes à linguagem porém cuja análise sintática foi bem sucedida graças às políticas de tratamento de erros implementada.

\section{Referências}
Foram utilizados basicamente os manuais do flex \cite{flex} e do bison \cite{bison}.
Tais fontes não eram de tão fácil compreensão, o que demandou esforços consideráveis em algumas situações.
% ---
% Finaliza a parte no bookmark do PDF, para que se inicie o bookmark na raiz
% ---
\bookmarksetup{startatroot}% 
% ---

% ---
% Conclusão
% ---
\postextual

\bibliography{bib}

\end{document}

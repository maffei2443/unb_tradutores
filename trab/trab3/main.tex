\documentclass[
	% -- opções da classe memoir --
	article,			% indica que é um artigo acadêmico
	11pt,				% tamanho da fonte
	oneside,			% para impressão apenas no recto. Oposto a twoside
	a4paper,			% tamanho do papel. 
	% -- opções da classe abntex2 --
	%chapter=TITLE,		% títulos de capítulos convertidos em letras maiúsculas
	%section=TITLE,		% títulos de seções convertidos em letras maiúsculas
	%subsection=TITLE,	% títulos de subseções convertidos em letras maiúsculas
	%subsubsection=TITLE % títulos de subsubseções convertidos em letras maiúsculas
	% -- opções do pacote babel --
	english,			% idioma adicional para hifenização
	brazil,				% o último idioma é o principal do documento
	sumario=tradicional
	]{abntex2}


% ---
% PACOTES
% ---

% ---
% Pacotes fundamentais 
% ---
\usepackage{lmodern}			% Usa a fonte Latin Modern
\usepackage[T1]{fontenc}		% Selecao de codigos de fonte.
\usepackage[utf8]{inputenc}		% Codificacao do documento (conversão automática dos acentos)
\usepackage{indentfirst}		% Indenta o primeiro parágrafo de cada seção.
\usepackage{nomencl} 			% Lista de simbolos
\usepackage{color}				% Controle das cores
\usepackage{graphicx}			% Inclusão de gráficos
\usepackage{microtype} 			% para melhorias de justificação
% ---
		
% ---
% Pacotes adicionais, usados apenas no âmbito do Modelo Canônico do abnteX2
% ---
\usepackage{lipsum}				% para geração de dummy text
% ---
		
% ---
% Pacotes de citações
% ---
\usepackage[brazilian,hyperpageref]{backref}	 % Paginas com as citações na bibl
\usepackage[alf]{abntex2cite}	% Citações padrão ABNT

\usepackage{backnaur} %  especificação da grmática
\usepackage{syntax} % especificação gramática
\usepackage{subfiles} % modularização em arquivos
\usepackage{hyperref} % hyper-links

%%%%%%%%%%%%%%%%%%%
%%%%%%%%%%%%%%%%%%% https://tex.stackexchange.com/questions/348651/c-code-to-add-in-the-document
%%%%%%%%%%%%%%%%%%%
\usepackage{xcolor}
\usepackage{listings}

\definecolor{mGreen}{rgb}{0,0.6,0}
\definecolor{mGray}{rgb}{0.5,0.5,0.5}
\definecolor{mPurple}{rgb}{0.58,0,0.82}
\definecolor{backgroundColour}{rgb}{0.95,0.95,0.92}

\lstdefinestyle{CStyle}{
	backgroundcolor=\color{backgroundColour},   
	commentstyle=\color{mGreen},
	keywordstyle=\color{magenta},
	numberstyle=\tiny\color{mGray},
	stringstyle=\color{mPurple},
	basicstyle=\footnotesize,
	breakatwhitespace=false,         
	breaklines=true,                 
	captionpos=b,                    
	keepspaces=true,                 
	numbers=left,                    
	numbersep=5pt,                  
	showspaces=false,                
	showstringspaces=false,
	showtabs=false,                  
	tabsize=2,
	language=C
}
%%%%%%%%%%%%%%%%%%%
%%%%%%%%%%%%%%%%%%%
%%%%%%%%%%%%%%%%%%%

% ---

% ---
% Configurações do pacote backref
% Usado sem a opção hyperpageref de backref

\renewcommand{\it}[1]{\textit{#1}}
\renewcommand{\bf}[1]{\textbf{#1}}

\renewcommand{\backrefpagesname}{Citado na(s) página(s):~}
% Texto padrão antes do número das páginas
\renewcommand{\backref}{}
% Define os textos da citação
\renewcommand*{\backrefalt}[4]{
	\ifcase #1 %
		Nenhuma citação no texto.%
	\or
		Citado na página #2.%
	\else
		Citado #1 vezes nas páginas #2.%
	\fi}%
% ---

% --- Informações de dados para CAPA e FOLHA DE ROSTO ---
\titulo{C-minus com operações matriciais}
%\tituloestrangeiro{Canonical article template in \abnTeX: optional foreign title}

\autor{
Leonardo Maffei da Silva\thanks{leoitu22hotmail.com@gmail.com. \url{https://www.linkedin.com/in/leonardo-maffei-ti/}} }

\local{Brasil}
\data{2019}
% ---

% ---
% Configurações de aparência do PDF final

% alterando o aspecto da cor azul
\definecolor{blue}{RGB}{41,5,195}

% informações do PDF
\makeatletter
\hypersetup{
     	%pagebackref=true,
		pdftitle={\@title}, 
		pdfauthor={\@author},
    	pdfsubject={Modelo de artigo científico com abnTeX2},
	    pdfcreator={LaTeX with abnTeX2},
		pdfkeywords={abnt}{latex}{abntex}{abntex2}{atigo científico}, 
		colorlinks=true,       		% false: boxed links; true: colored links
    	linkcolor=blue,          	% color of internal links
    	citecolor=blue,        		% color of links to bibliography
    	filecolor=magenta,      		% color of file links
		urlcolor=blue,
		bookmarksdepth=4
}
\makeatother
% --- 

% ---
% compila o indice
% ---
\makeindex
% ---

% ---
% Altera as margens padrões
% ---
\setlrmarginsandblock{3cm}{3cm}{*}
\setulmarginsandblock{3cm}{3cm}{*}
\checkandfixthelayout
% ---

% --- 
% Espaçamentos entre linhas e parágrafos 
% --- 

% O tamanho do parágrafo é dado por:
\setlength{\parindent}{1.3cm}

% Controle do espaçamento entre um parágrafo e outro:
\setlength{\parskip}{0.2cm}  % tente também \onelineskip

% Espaçamento simples
\SingleSpacing


% ----
% Início do documento
% ----
\begin{document}

% Seleciona o idioma do documento (conforme pacotes do babel)
%\selectlanguage{english}
\selectlanguage{brazil}

% Retira espaço extra obsoleto entre as frases.
\frenchspacing 

% ----------------------------------------------------------
% ELEMENTOS PRÉ-TEXTUAIS
% ----------------------------------------------------------

%---
%
% Se desejar escrever o artigo em duas colunas, descomente a linha abaixo
% e a linha com o texto ``FIM DE ARTIGO EM DUAS COLUNAS''.
% \twocolumn[    		% INICIO DE ARTIGO EM DUAS COLUNAS
%
%---

% página de titulo principal (obrigatório)
\maketitle


% titulo em outro idioma (opcional)



% resumo em português
\begin{resumoumacoluna}
 Este documento atende os fins de documentação da segunda
 parte do projeto final da disciplina \textit{Tradutores},
 ministrada pela professora Dr.a Cláudia Nalon, no segundo semestre de 2019, na Universidade de Brasília. Tal artefato descreve um pouco da implementação do analisador léxico, dificuldades encontradas durante tal processo; a gramática da linguagem proposta na parte 1 deste trabalho, bem como as alterações nela realizada; políticas de tratamento de erros e arquivos de teste para o analisador produzido.
  \cite{gramatica}.
 \vspace{\onelineskip}
 
 \noindent
 \textbf{Palavras-chave}: C, linguagem, matriz, primitiva.
\end{resumoumacoluna}




\newcommand{\terminal}[1]{ \bnfpn{\textbf{#1}} }

\newcommand{\production}[1]{\bnfpn{\textit{#1}}}
\newcommand{\IT}[1]{\textit{#1}}

%\setlength{\grammarparsep}{20pt plus 1pt minus 1pt} % increase separation between rules
%\setlength{\grammarindent}{12em} % increase separation between LHS/RHS 

\section{Introdução}
Implementar-se-á, até a versão final deste artigo, um compilador para a linguagem proposta. Para sua realização, serão utilizados os conhecimentos adquiridos na disciplina \textit{Tradutores}, ministrada pela professora \hyperref{http://lattes.cnpq.br/7793795625581127}{}{}{Cláudia Nalon.}

\section{Usuário característico}
Destina-se ao estudante de álgebra linear, o qual pode usar a linguagem para, por exmeplo, confirmar se sua resolução de um sisema linear encontra-se correta, tudo isso de maneira rápida, eficiente e \it{offline}.

\section{Motivação}
Durante a realização do curso de Cálculo Numérico, o grupo do autor notou a ausência dessa \it{feature} na linguagem C. Desse modo, foi necessária a simulação desse tipo de dados, à época implementada por meio de inúmeras funções. Se houvese um tipo nativo para matriz bem como operações elementares sobre seus elementos, teria sido de grande auxílio à codificação dos diversos métodos numéricos requeridos pela disciplina.
\par


\section{Gramática}
A seguir, encontra-se a gramática da linguagem proposta:
%% BEGIN GRAMÁTICA
	\subfile{grammar.tex}
%% END GRMÁTICA
Em relação à versão pregressa passado, note-se as principais mudanças:
\begin{itemize}{
	\item adição dos operadores \IT{@} e \IT{@@}, sendo respectivamente o operador de multiplicação e potenciação de matrizes.
	\item adição do tipo de dado \IT{char}, para posibilitar uma capacidade de comunicação com o usuário superior à que havia antes, a qual proporcionava apenas tipos de dados numéricos.
	\item adição de \IT{strings}, tendo em vista a interação com o usuário (semelhante à adição do \IT{char}). Seu destaque fica por conta da simplicidade em se mostrar mensagens "de uma só vez" quando comparado a imprimir um caractere por comando. Limitação: o usuário apenas pode trabalhar com \IT{strings} \textbf{literais}, e não existe o tipo de dados \IT{string}.
	\item removeu-se a palavra reservada \textbf{void} da linguagem. Desse modo, todas as funções passam a retornar algum valor e não é mais possível a declaração de funções cujo único conteúdo entre os delimitadores de parâmetros era esta palavra reservada. Onde havia ocorrência daquela foi deixada a cadeia vazia.
	\textbf{void}. A vantagem é a simplificação da gramática bem como remoção de um tipo de função "não essencial" ao usuário.
	\item sutil alteração na regra \IT{<param>}, tal que agora é gramaticalmente inválido a declaração de um vetor de matrizes(erro semântico, mas que em versão anterior da gramática era sintaticamente válido).
	\item Além das mudanças acima citadas, foram realizados algumas renomeações de regras afim de representação mais compacta da gramática e desse modo requerindo menos esforço visual para perceber todas as suas regras bem como a relação entre elas. As mudanças de nome realizadas foram:
		\begin{itemize} {
			\item \IT{declaration} $\rightarrow$ \IT{decl}
			\item \IT{specifier} $\rightarrow$ \IT{spec}
			\item \IT{declarations} $\rightarrow$ \IT{decls}
			\item \IT{statement} $\rightarrow$ \IT{stmt}
			\item \IT{iteration} $\rightarrow$ \IT{while}
			\item \IT{int-nested-seq} $\rightarrow$ \IT{int-seq-list}		
			\item \IT{float-nested-seq} $\rightarrow$ \IT{int-seq-list}
			\item \IT{expression} $\rightarrow$ \IT{expr}
		}\end{itemize}
	\item por fim, há a alteração do lado direito da regra \IT{<arg-list>}, substituindo \IT{<expression>} por \IT{<simple-expr>}, visto que de fato não faz sentido permitir operações de atribuição na passagem de parâmetros para funções.
}\end{itemize}
% ----------------------------------------------------------
% Introdução
% ----------------------------------------------------------


% ----------------------------------------------------------
% Seção de explicações
% ----------------------------------------------------------




\section{Semântica}
A semântica da linguagem é quase semelhante à da linguagem C: declarações de variáveis (a menos das do tipo \bf{mat}), funções e expressões têm semântica similar. Sendo esta linguagem uma extensão de um subconjunto da linguagem \it{C}, a principal diferença está no tipo de dados \bf{mat} (abreviação de \it{matrix}). Esse tipo de dados é similar aos \it{arrays} em C, porém limitado do ponto de vista da composição pois não é possível a criação de matrizes aninhadas, nem de matrizes de vetores. Entretando, é possível a realização das quatro operações aritméticas básicas diretamente com matrizes, bem como a realização de potenciação de matrizes de forma \it{rápida} e algumas operações sobre elas, como resolução de sistemas lineares e escalonamento. Multiplicação e potenciação de matrizes são respectivamente expressas pelos novos operadores \IT{@} e \IT{@@}.

\section{Exemplo de programa na linguagem}
A seguir, trechos de código pertencente à nova linguagem.

\begin{lstlisting}[style=CStyle]
int main() {
float a = 10.1;
float c = 10.;
float d = .1;
float b = .29;

mat<int> m[3][3] = {{1, 0, 0}, {0 ,1, 0}, {0, 0, 1}};
scan(a);
}'a'; '\n'; '\r'; '\\';
\end{lstlisting}

\begin{lstlisting}[style=CStyle]
float main;
{1,2,3};
"Pode ir, tudo bem..."
/* Lucero mto bom */
print('h');
print('e');
print('l');
print('l');
\end{lstlisting}

\section{Exemplo de programa não pertencentes à linguagem}

\begin{lstlisting}[style=CStyle]
" string sem dim!

int main() $$ {
;;;
}
\end{lstlisting}

\begin{lstlisting}[style=CStyle]
float chr(void);
'\\\\';
/*
int main() {

}
print("e agora, joseh?")
Obs: a exemplo do tratamento proporcionado pelo gcc,
nao eh feito nenhuma tentativa de recuperacao para
erros de comentarios sem fechamento ;)
\end{lstlisting}



\section{Implementação}
O presente analisador léxico tem como dependências, além do programa \IT{flex}, os respectivos arquivos fonte, além de seus respectivos cabeçalhos (com exceção do último, por tratar do código fonte utilizado pelo \IT{flex} para geração do arquivo \textbf{lex.yy.c}):
	\begin{itemize}{
		\item{Array.c}
		\item{Colorfy.c}
		\item{SymTable.c}
		\item{leo.l}
	}\end{itemize}
Todos os arquivos estão disponíveis publicamente \href{https://github.com/maffei2443/unb_tradutores/tree/master/trab/trab2/code/src}{neste \IT{link}}, bem como o \IT{makefile} utilizado para compilação do analisador. Contudo, \textbf{recomenda-se fortemente} a execução das seguintes instruções, na ordem em que aparecem, para garantia da correta geração do analisador:

\begin{lstlisting}[language=bash,caption={bash version}]
	make clean
	flex leo.l
	make
	./lexico <caminho-para-arquivo>
\end{lstlisting}
As instruções além do clássico \IT{make} fazem-se necessárias pois não foi conseguido pelo autor a automação do processo de transpilação do código em \IT{leo.l} para \IT{lex.yy.c} seguido da compilação e ligação entre os arquivos necessários à geração do analisador de forma automática \IT{com a garantia de nova geração do \textbf{lex.yy.c} sempre que fosse dado o comando \textbf{make}}. Não foi necessária nenhuma modificação direta do arquivo \textbf{lex.yy.c} em momento algum do desenvolvimento.

\subsection{Funcionamento}
Após gerado o analisador léxico seguindo os passos descritos no início desta seção, sua utilização é bastante simples: execute o programa passando como argumento o caminho para o arquivo que deve ser aberto e processado pelo léxico. O programa então lê sequenciamente o arquivo caractere a caractere e vai exibindo os \textit{tokens} presentes no arquivo apontado à medida que os encontra, bem como trata os erros descritos na subseção logo abaixo.
A saída é exibida na saída padrão (usualmente, um console) e de forma \textit{colorida}. Entende-se que isso facilita a visualização por parte dos humanod; contudo, isso deve ser adaptado na próxima fase para que seja gerado um arquivo estruturado contendo todos os tokens identificados, o qual será utilizado pelo analisador sintático.

\subsection{Tratamento de Erros}
O analisador léxico desenvolvido é capaz de detectar os seguintes erros:
\begin{enumerate}{
	\item \label{itm:er1}\IT{string} sem fechamento
	\item \label{itm:er2}comentários em bloco sem fechamento
	\item \label{itm:er3}caracteres não pertencentes à linguagem (individualmente)
}\end{enumerate}
O erro \ref{itm:er1} é tratado considerando que o usuário termina a \IT{string} ao final da linha, visto que não são permitidas \IT{strings} multi-linhas, e continuando a análise como se não houvesse erro.
O próximo erro (\ref{itm:er2}) não é exatamente tratado; a abordagem utilizada é a mesma do compilador \href{https://gcc.gnu.org/}{gcc 7.7.0}: é emitido um aviso ao usuário informando-lhe linha e coluna onde se inicia o comentário não finalizado.
Por fim, o último erro (\ref{itm:er3}) é tratado informando ao usuário as ocorrências desses caracteres, porém sem entrar em modo de pânico.

\subsection {Dificuldades Encontradas}
Uma das dificuldades foi sem dúvida a familiarização com a ferramente \IT{flex}, a qual demonstrou-se muito competente porém não tão intuitiva, bem como alguns trechos de seu \href{ftp://ftp.gnu.org/old-gnu/Manuals/flex-2.5.4/html_mono/flex.html}{manual}. Além disso, surgiram dúvidas a respeito do que seriam erros léxicos, como por exemplo qual o escopo dessa categoria de erros.

Além das dúvidas acima, foram encontrados obstáculos na automação da geração do analisador por meio da ferramenta \IT{make}, questões sobre como deveria ser exibida a sequência de \IT{tokens} (quais possíveis metadados deveriam ser exebidos?) e algumas pontualidades acerca de recursão na gramática.

Em termos de codificação, além do problema recém-citado, não foi possível implementar a tabela de símbolos a tempo da entrega (ocorreram problemas de vazamento de memória e alocação equivocada). Embora esta não seja fundamental para a exibição dos tokens lidos em sequência, sua presença certamente enriqueceria o presente trabalho.

\subsection{Arquivos de teste}
Os arquivos de teste encontram-se disponíveis \href{https://github.com/maffei2443/unb_tradutores/tree/master/trab/trab2/code/test}{neste link}. Os arquivos desprovidos de erros léxicos têm seus nomes iniciados pela letra 'c', ao passo que os demais iniciam com a letra 'e'.

\section{Agradecimentos}
Agradecimentos ao responsável por \cite{gramatica}, sem o qual este trabalho teria sido muito mais custoso, bem como ao usuário \hyperref{https://tex.stackexchange.com/users/42366/croco}{}{}{CroCro}, o qual possibilitou conforme \cite{custom} rápida customização do código-exemplo da nova linguagem. Por fim, o pacote utilizado para redação da gramática foi sugerido pelo usuário \hyperref{https://tex.stackexchange.com/questions/348651/c-code-to-add-in-the-document}{}{}{AlanMunn}, conforme \cite{bnf}.

% ---
% Finaliza a parte no bookmark do PDF, para que se inicie o bookmark na raiz
% ---
\bookmarksetup{startatroot}% 
% ---

% ---
% Conclusão
% ---
\postextual

\bibliography{bib}

\end{document}
